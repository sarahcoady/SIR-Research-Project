\documentclass{article}
\usepackage{graphicx} % Required for inserting images

\title{SIR Research Proposal}
\author{Sarah Coady}
\date{January 2026}

\begin{document}

\maketitle

\section{Introduction}
\newpage
Epidemic modelling (EM) is an important tool  in epidemiology using various mathematical tools to help study and predict the spread of real world infectious diseases (NLM). There is a wide range of epidemic models each using different assumptions about population and transmission. This project will study the Susceptible, infected, Removed (SIR) model. Specifically, we will test how accurate the SIR model is at producing real world infectious disease outbreak patterns across two phases, the early growth phase and the mid stage, close to its peak. This will be accomplished by comparing curves acquired from the SIR model to real historical data. By applying the SIR model to real outbreaks we can examine where it successfully replicates real data and where it fails. Understanding the spread of infectious disease will always be highly important. Thus, it is important to have a strong understanding of this model. Knowing its strengths and limitations allows for correct application. We will apply the SIR model to the historical records of both Measles and COVID-19 as these were two highly contagious and historically impactful infectious diseases.

\section{The SIR Model}
The SIR model is a set of three ordinary differential equations (ODEs) that can be used to obtain a stronger understanding of the spread of infectious diseases. The model divides a population into three groups.

\textbf{Susceptible individuals, S(t)}: People in the population that are not infected but could become infected (SD). 

\textbf{Infected Individuals, I(t)}: People in the Population who have been infected by the disease and can ten transmit it to other susceptible individuals.

\textbf{Removed individuals, R(t)}: People in the population who have recovered from the diseased or who have died.

The SIR model then uses basic assumptions and the rate of change of each of the three groups to acquire the system ODEs. 

\section{Research Questions}

We want to examine if the SIR model can accuartely describe the early exponential growth stage, correctly predict the peak of the spread and if the model performs better or worse in the early stage compared to the middle.

\section{Methods}


\end{document}

